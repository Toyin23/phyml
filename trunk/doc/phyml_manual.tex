\documentclass[a4paper,12pt]{article}
\usepackage{fancyvrb}
\usepackage{graphicx}
\usepackage{tabularx}
\usepackage{color}
\usepackage{psfrag}
\usepackage{url}
\usepackage{vmargin}
\usepackage{cite}
\usepackage{caption2}
\usepackage{hyperref}
\usepackage{makeidx}
\renewcommand{\captionlabeldelim}{.}
\def\thesection{\arabic{section}}
\renewcommand{\thefigure}{\arabic{figure}}
\renewcommand{\thetable}{\arabic{table}}
\renewcommand{\baselinestretch}{1.5}
\newcommand{\hl}{\noalign{\vskip3pt}\hline\noalign{\vskip3pt}}
\newcommand{\hrf}{\hrulefill}
\newcommand{\tc}[1]{\textcolor{black}{#1}}
\newcommand{\dc}[1]{\textcolor{green}{#1}}
% \usepackage[none]{hyphenat}
\newcommand{\rep}[3][1]
{
  \psfrag{#2}[c][c][#1]{#3}
}
\newcommand{\x}[1]{\texttt{#1}}

\setpapersize{A4}
\hypersetup{colorlinks=true,linkcolor=blue,urlcolor=red,linkbordercolor=000}
\renewcommand{\baselinestretch}{1}
\makeindex
\begin{document}
\begin{center}
\thispagestyle{empty}
\vfill\vfill
{\Huge \textbf{ PhyML~~--~~Manual}}
\vfill
{\huge Version 3.0 \\
\today
\vfill
\href{http://www.atgc-montpellier.fr/phyml}{http://www.atgc-montpellier.fr/phyml}}
\end{center}
\clearpage
\tableofcontents
\clearpage

{\par
\small 
\noindent
\copyright Copyright 1999 - 2008 by PhyML Development Team.\\
\noindent The software PhyML is provided ``as is''  without warranty of any kind.  In no event shall
the authors  or his  employer be  held responsible  for any damage  resulting from  the use  of this
software, including but not limited to the frustration that you may experience in using the package.
All parts of the source and documentation except where indicated are distributed under
the GNU public licence. See http://www.opensource.org for details.

}

{
\noindent
\setlength{\baselineskip}{0.5\baselineskip}
\section{Citation}
\begin{itemize}
\item ``A simple, fast and accurate algorithm to estimate large phylogenies by maximum likelihood''
  Guindon S., Gascuel O. {\it Systematic Biology} {\bf \small 52}(5):696-704
\end{itemize}
}

{
\noindent
\setlength{\baselineskip}{0.7\baselineskip}
\section{Authors}
\begin{itemize}
\item { St\'ephane Guindon} and { Olivier Gascuel} conceived the original PhyML algorithm.
\item { St\'ephane Guindon, Wim Hordjik} and { Olivier Gascuel} conceived the SPR-based tree search algorithm.
\item { Maria Anisimova} and { Olivier Gascuel} conceived the aLRT method for branch support.
\item { St\'ephane Guindon, Franck Lethiec}, Jean-Francois Dufayard and Vincent Lefort implemented PhyML.
\item { Jean-Francois Dufayard} created the benchmark and implemented the tools that are used to check
  PhyML accuracy and performances. 
\item { Vincent Lefort, St\'ephane Guindon, Patrice Duroux} and { Olivier Gascuel} conceived and
  implemented PhyML web server.
\item St\'ephane Guindon wrote this document.
\end{itemize}
}
\clearpage

% \section{ML in phylogenetics: the basics.}
\section{Overview}

PhyML  \cite{guindon03} is  a  software  package which  primary  task that  is  to estimate  maximum
likelihood phylogenies  from alignments of nucleotide or  amino acid sequences.  It  provides a wide
range  of  options that  were  designed  to facilitate  standard  phylogenetic  analyses.  The  main
strengths of  PhyML lies in the  large number of substitution  models coupled to  various options to
search the  space of phylogenetic  tree topologies,  going from very  fast and efficient  methods to
slower but  generally more accurate approaches.  It  also implements two methods  to evaluate branch
supports  in  a  sound statistical  framework  (the  non-parametric  bootstrap and  the  approximate
likelihood ratio test,)

PhyML was designed to  process moderate to large data sets.  In theory,  alignments with up to 4,000
sequences 2,000,000 character-long can analyzed.  In practice however, the amount of memory required
to process  a data set is proportional  of the product of  the number of sequences  by their length.
Hence, a large number  of sequences can only be processed provided that  they are short. Also, PhyML
can  handle long  sequences  provided  that they  are  not numerous.   With  most standard  personal
computers, the  ``comfort zone'' for PhyML generally  lies around 100-200 sequences  less than 2,000
character  long.   For  larger  data  sets,  we  recommend using  other  software's  such  as  RAxML
\cite{raxml}\index{RAxML}      or       GARLI      \cite{garli}\index{GARLI}      or      Treefinder
(\href{http://www.treefinder.de}{http://www.treefinder.de}).


\section{Installing PhyML}

\subsection{Sources and compilation}\index{compilation}

The sources of the  program are available free of charge by sending  an e-mail to St\'ephane Guindon
at                      \href{mailto:guindon@lirmm.fr}{\x{guindon@lirmm.fr}}                      or
\href{mailto:guindon@stat.auckland.ac.nz}{\x{guindon@stat.auckland.ac.nz}}.

The compilation on UNIX-like systems is fairly  standard. It is described in the `INSTALL' file that
comes with the sources. In a command-line window,  go to the directory that contains the sources and
type:

{\setlength{\baselineskip}{0.5\baselineskip}
\begin{verbatim}
./configure;
make clean;
make;
\end{verbatim}
}

{\em Note} -- when PhyML  is going to be used mostly of exclusively in  batch mode, it is preferable
to turn on the batch mode option in the  Makefile. In order to do so, the file \x{Makefile.am} needs
to be modified: add \x{-DBATCH} to the line with \x{DEFS=-DUNIX -D\$(PROG) -DDEBUG}.



\subsection{Installing PhyML on UNIX-like systems (including Mac OS)}

Copy PhyML binary file in the directory you like.  For the operating system to be able to locate the
program, this directory must be specified in the global variable \x{PATH}. In order to achieve this,
you  will   have  to  add  \x{export   PATH="/your\_path/:\${PATH}"}  to  the   \x{.bashrc}  or  the
\x{.bash\_profile} located in your home directory  (\x{your\_path} is the path to the directory that
contains PhyML binary).


\subsection{Installing PhyML on Microsoft Windows}\label{sec:install_windows}

Copy the files \x{phyml.exe} and \x{phyml.bat} in  the same directory. To launch PhyML, click on the
icon  corresponding to \x{phyml.bat}.   Clicking on  the icon  for \x{phyml.exe}  works too  but the
dimensions of the window will not fit PhyML interface.

\subsection{Installing the parallel version of PhyML}\label{sec:MPI}\index{MPI}\index{bootstrap!parallel}

Bootstrap analysis can run on multiple  processors. Each processor analyses one bootstraped dataset.
Therefore, the computing time needed to perform $R$ bootstrap replicates is divided by the number of
processors available.

This  feature of  PhyML relies  on the  MPI (Message  Passing Interface)  library. To  use  it, your
computer must  have MPI  installed on  it. In case  MPI is  not installed, you  can dowload  it from
\href{http://www.mcs.anl.gov/research/projects/mpich2/}{http://www.mcs.anl.gov/research/projects/mpich2/}.
Once  MPI is  installed, a  few modification  of the  file `\x{Makefile.am}'  must be  applied.  The
relevant section  of this file  and the  instruction to add  or remove the  MPI option to  PhyML are
printed below:

\begin{verbatim}
# Uncomment (i.e. remove the `#' character at the begining of) 
# the two lines below if you want to use MPI.
# Comment the two lines below if you don't want to use MPI.

# CC=mpicc
# DEFS=-DUNIX -D$(PROG) -DDEBUG -DMPI

# Comment the line below if you want to use MPI.
# Uncomment the line below if you don't want to use MPI.

DEFS=-DUNIX -D$(PROG) -DDEBUG
\end{verbatim}


\section{Program usage.}\label{sec:phyml_new}

PhyML  has two distinct  user-interfaces.  The  first interface  is probably  the most  popular.  It
corresponds to  a PHYLIP-like text interface that  makes the choice of  the options self-explanatory
(see  Figure \ref{fig:interface}). The  command-line interface  is well-suited  for people  that are
familiar with PhyML options or for running PhyML in batch mode.

\subsection{PHYLIP-like interface}

The default is  to use the PHYLIP-like text interface (Figure  \ref{fig:interface}) by simply typing
`\x{phyml}'  in   a  command-line   window  or  by   clicking  on   the  PhyML  icon   (see  Section
\ref{sec:install_windows}).  After entering the name of the input sequence file, a list of sub-menus
helps the users to set up the analysis.  There are currently four distinct sub-menus:

\begin{figure}
\resizebox{15cm}{9cm}{\includegraphics{./fig/interface.eps}}
\caption{PHYLIP-like interface to PhyML.}
\label{fig:interface}
\end{figure}

\begin{enumerate}

\item  {\em  Input  Data}:  specify  whether  the  input  file  contains  amino-acid  or  nucleotide
sequences. What the  sequence format is (see Section \ref{sec:input_output}) and  how many data sets
should be analysed.

\item  {\em  Substitution  Model}: selection  of  the  Markov  model  of substitution.

\item  {\em  Tree Searching}:  selection  of  the tree  topology  searching  algorithm.

\item {\em  Branch Support}: selection  of the method  that is used  to measure branch  support.

\end{enumerate}
\noindent `\x{+}' and `\x{-}' keys are used to  move forward and backward in the sub-menu list. Once
the model parameters  have been defined, typing `\x{Y}' (or `\x{y}')  launches the calculations. The
meaning of  some options may not  be obvious to users  that are not familiar  with phylogenetics. In
such situation, we strongly recommend to use the default options. As long as the format of the input
sequence file is  correctly specified (sub-menu {\em Input data}), the  safest option for non-expert
users is to use the default settings.

The different options provided within each sub-menu are described in what follows.


\subsubsection{Input  Data sub-menu}

\begin{center}\framebox{\x{[D] ............................... Data type (DNA/AA)}} \end{center} 
Type of data in the input file. It can be either DNA or amino-acid sequences in PHYLIP format (see
Section \ref{sec:input_output}). Type \x{D} to change settings.

\vspace{0.7cm}
\begin{center} \framebox{\x{[I] ...... Input sequences interleaved (or sequential)}} \end{center}
PHYLIP format comes in two flavours: interleaved or sequential (see Section
\ref{sec:input_output}). Type \x{I} to selected among the two formats.

\vspace{0.7cm}
\begin{center} \framebox{\x{[M] ....................... Analyze multiple data sets}} \end{center}
If the input sequence file contains more than one data sets, PhyML can analyse each of them
in a single run of the program. Type \x{M} to change settings.

\vspace{0.7cm}
\begin{center}  \framebox{\x{[R] ............................................ Run  ID}} \end{center}
This option allows  you to append a string  that identifies the current PhyML run.  Say for instance
that you want to analyse  the same data set with two models. You can  then `tag' the first PhyML run
with the name of the first model while the second run is tagged with the name of the second model.\index{run ID}



\subsubsection{Substitution model sub-menu}\label{sec:submenus}

\begin{center} \framebox{\x{[M] ................. Model of nucleotide substitution}} \end{center}
\begin{center}  \framebox{\x{[M] ................ Model  of amino-acids  substitution}} \end{center}
PhyML implements a wide range of  substitution models: JC69 \cite{jukes69}, K80 \cite{kimura80}, F81
\cite{felsenstein81a},  F84  \cite{phylip2},   HKY85  \cite{hasegawa85},  TN93  \cite{tamura93}  GTR
\cite{lanave84,tavare86} and  custom for nucleotides; LG \cite{le08},  WAG \cite{whelan01b}, Dayhoff
\cite{dayhoff78},  JTT  \cite{jones92},  Blosum62  \cite{henikoff92}, mtREV  \cite{adachi96},  rtREV
\cite{dimmic02},  cpREV  \cite{adachi00},  DCMut   \cite{kosiol04},  VT  \cite{muller00}  and  mtMAM
\cite{cao98}  anf custom  for amino  acids.  Cycle  through the  list of  nucleotide  or amino-acids
substitution models by typing \x{M}. Both  nucleotide and amino-acid lists include a `custom' model.
The custom option provides the most flexible  way to specify the nucleotide substitution model.  The
model is defined  by a string made of  six digits.  The default string is  `\x{000000}', which means
that the six relative rates of nucleotide  changes: $A \leftrightarrow C$, $A \leftrightarrow G$, $A
\leftrightarrow T$,  $C \leftrightarrow  G$, $C  \leftrightarrow T$ and  $G \leftrightarrow  T$, are
equal.   The   string  `\x{010010}'  indicates  that   the  rates  $A  \leftrightarrow   G$  and  $C
\leftrightarrow  T$ are equal  and distinct  from $A  \leftrightarrow C  = A  \leftrightarrow T  = C
\leftrightarrow G = G  \leftrightarrow T$.  This model corresponds to HKY85  (default) or K80 if the
nucleotide frequencies  are all set to 0.25.   `\x{010020}' and `\x{012345}' correspond  to TN93 and
GTR models respectively.  The digit string  therefore defines groups of relative substitution rates.
The  initial rate within  each group  is set  to 1.0,  which corresponds  to F81  (JC69 if  the base
frequencies are equal).   Users also have the  opportunity to define their own  initial rate values.
These rates are  then optimised afterwards (option  `\x{O}') or fixed to their  initial values.  The
custom option can be used to implement all substitution models that are special cases of GTR.

The custom model also exists for protein sequences. It is useful when one wants to use an amino-acid
substitution model that is  not hard-coded in PhyML. The symmetric part of  the rate matrix, as well
as the equilibrium amino-acid  frequencies, are given in a file which name is  given as input of the
program. The format of this file is described in the section \ref{sec:customaa}.

\vspace{0.7cm}
\begin{center} \framebox{\x{[F] ................. Optimise equilibrium frequencies}} \end{center}
\begin{center} \framebox{\x{[E] ......... Equilibrium frequencies (empirical/user)}} \end{center}
\begin{center} \framebox{\x{[F]  . Amino  acid frequencies (empirical/model  defined)}} \end{center}
For  nucleotide  sequences,  optimising  nucleotide  frequencies  means that  the  values  of  these
parameters  are estimated in  the maximum  likelihood framework.   When the  custom model  option is
selected, it is also possible to give the program a user-defined nucleotide frequency distribution
at  equilibrium (option \x{E}).  For protein  sequences, the  stationary amino-acid  frequencies are
either  those defined  by  the substitution  model  or those  estimated by  counting  the number  of
different amino-acids observed  in the data. Hence, users  should be well aware that  the meaning of
the \x{F} option depends on the type of the data to be processed.

\vspace{0.7cm}
\begin{center}          \framebox{\x{[T]          ....................          Ts/tv          ratio
(fixed/estimated)}}   \end{center}\index{$\kappa$}\index{ts/tv   ratio}    Fix   or   estimate   the
transition/transversion ratio  in the maximum likelihood  framework.  This option  is only available
when DNA sequences are to be analysed under  K80, HKY85 or TN93 models. The definition given to this
parameter by PhyML  is the same as  PAML's\index{PAML} one.  Therefore, the value  of this parameter
does {\it not} correspond  to the ratio between the expected number  of transitions and the expected
number  of  transversions  during  a unit  of  time.   This  last  definition  is the  one  used  in
PHYLIP\index{PHYLIP}.   PAML's  manual gives  more  detail about  the  distinction  between the  two
definitions.

\vspace{0.7cm}
\begin{center}       \framebox{\x{[V]       .        Proportion      of       invariable       sites
(fixed/estimated)}}   \end{center}\index{invariable  sites}\index{proportion   of   invariants}  The
proportion of  invariable sites, i.e., the  expected frequency of sites  that do not  evolve, can be
fixed or estimated. The default is to fix this proportion to 0.0. By doing so, we consider that each
site in  the sequence may accumulate  substitutions at some point  during its evolution,  even if no
differences across sequences are actually observed at  that site.  Users can also fix this parameter
to  any value  in the  $[0.0,1.0]$ range  or estimate  it from  the data  in  the maximum-likelihood
framework.

\vspace{0.7cm} 
\index{gamma distribution (discrete)!mean vs. median}
\index{gamma distribution (discrete)!number of categories}
\index{gamma distribution (discrete)!shape parameter}
\begin{center} \framebox{\x{[R]  ....... One category  of substitution rate  (yes/no)}} \end{center}
\begin{center} \framebox{\x{[C] ........... Number of substitution rate categories}} \end{center}
\begin{center} \framebox{\x{[A] ... Gamma distribution parameter (fixed/estimated)}} \end{center}
\begin{center} \framebox{\x{[G] .........`Middle' of each rate class (mean/median)}} \end{center}
Rates of evolution often vary from site to site. This heterogeneity can be modelled using a discrete
gamma distribution. Type \x{R} to switch this option on or off.

The different categories  of this discrete distribution correspond to  different (relative) rates of
evolution. The number of categories of this distribution is set to 4 by default.  It is probably not
wise to  go below this  number.  Larger values  are generally preferred. However,  the computational
burden involved  is proportional to the  number of categories  (i.e., an analysis with  8 categories
will generally  take twice  the time of  the same analysis  with only  4 categories). Note  that the
likelihood will not necessarily increase as the number of categories increases. Hence, the number of
categories should be kept  below a ``reasonable'' number, say 20.  The  default number of categories
can be changed by typing \x{C}.  

The middle  of each  discretized substitution rate  class can  be determined using  the mean  or the
median. PAML,  MrBayes and RAxML  use the  mean.  However, the  median is generally  associated with
greater likelihoods than the median.  This conclusion is based on our analysis of several real-world
data sets extracted from TreeBase.  Despite this, the  default option in PhyML is to use the mean in
order to make PhyML  likelihoods comparable to those of other phylogenetic  software.  One must bare
in  mind that  {\color{red}{likelihoods  calculated with  the  mean approximation  are not  directly
comparable to the likelihoods calculated using the median approximation}}.

The shape  of the  gamma distribution  determines the range  of rate  variation across  sites. Small
values,  typically  in  the $[0.1,1.0]$  range,  correspond  to  large variability.   Larger  values
correspond to moderate to  low heterogeneity. The gamma shape parameter can be  fixed by the user or
estimated via maximum-likelihood. Type \x{A} to select one or the other option.





\subsubsection{Tree searching sub-menu}

\begin{center} \framebox{\x{[O] ........................... Optimise tree topology}} \end{center} By
default the  tree topology is  optimised in order  to maximise the  likelihood. However, it  is also
possible to avoid  any topological alteration. This option  is useful when one wants  to compute the
likelihood of a tree given as input (see below). Type \x{O} to select among these two options.

\vspace{0.7cm}
\begin{center}  \framebox{\x{[S] .................. Tree  topology search  operations}} \end{center}
PhyML proposes three different  methods to estimate tree topologies. The default  approach is to use
simultaneous  NNI. This option  corresponds to  the original  PhyML algorithm  \cite{guindon03}. The
second approach  relies on  subtree pruning and  regrafting (SPR).   It generally finds  better tree
topologies  compared to NNI  but is  also significantly  slower.  The  third approach,  termed BEST,
simply estimates the phylogeny using both methods  and returns the best solution among the two. Type
\x{S} to choose among these three choices.

\vspace{0.7cm}
\begin{center}  \framebox{\x{[R] ......................... Use  random starting  tree}} \end{center}
\begin{center} \framebox{\x{[N]  .................. Number  of random starting  trees}} \end{center}
When the SPR or the BEST options are selected,  is is possible to use random trees rather than BioNJ
or a user-defined tree,  as starting tree. If this option is turned on  (type \x{R} to change), five
trees, corresponding to five random starts, will be estimated. The output tree file will contain the
best tree found among those five. The number of random starts can be modified by typing \x{N}.

\vspace{0.7cm}
\begin{center}      \framebox{\x{[U]     ........       Starting      tree     (BioNJ/parsimony/user
tree)}} \end{center}\index{BioNJ}  When the  tree topology optimisation  option is turned  on, PhyML
proceeds  by  refining an  input  tree.   By  default, this  input  tree  is estimated  using  BioNJ
\cite{gascuelNJ}. The alternative option is to use  a parsimony tree. We found this option specially
useful when analysing  large data sets with NNI  moves as it generally leads  to greater likelihoods
than  those obtained  when starting  from a  BioNJ trees.  The user  can also  to input  her/his own
tree. This  tree should  be in Newick  format (see  Section \ref{sec:input_output}). This  option is
useful when  one wants  to evaluate  the likelihood  of a given  tree with  a fixed  topology, using
PhyML. Type \x{U} to choose among these two options.

\subsubsection{Branch support sub-menu}

\begin{center}  \framebox{\x{[B] ................ Non  parametric bootstrap  analysis}} \end{center}
The  support  of the  data  for  each  internal branch  of  the  phylogeny  can be  estimated  using
non-parametric bootstrap.   By default, this option is  switched off.  Typing \x{B}  switches on the
bootstrap analysis. The user is then prompted for a number of bootstrap replicates. The largest this
number  the more  precisely  the bootstrap  support are.  However,  for each  bootstrap replicate  a
phylogeny is  estimated. Hence, the time needed  to analyse $N$ bootstrap  replicates corresponds to
$N$-times the time spent  on the analysis of the original data  set. $N=100$ is generally considered
as a reasonable number of replicates.

\begin{center}  \framebox{\x{[A] ................ Approximate  likelihood ratio  test}} \end{center}
When the  bootstrap option is switched off  (see above), approximate likelihood  branch supports are
estimated. This approach is considerably faster than the bootstrap one. However, both methods intend
to  estimate different quantities  and conducting  a fair  comparison between  both criteria  is not
straightforward. The estimation of approximate likelihood  branch support comes in two flavours: the
measured  statistics  is  compared to  a  $\chi^2$  distribution  or a  non-parametric  distribution
estimated using a RELL approximation.



\subsection{Command-line interface}

The alternative to the  PHYLIP-like interface is the command line. Users that  do not need to modify
the default parameters can launch the  program with the `\x{phyml -i seq\_file\_name}' command.  The
list of  all command line  arguments and how  to use them  is given in  the `Help' section  which is
displayed  after entering the  `\x{phyml help}'  command.  The  options are  also described  in what
follows.

\begin{itemize}
\item \x{-i} (or \x{--input}) \x{seq\_file\_name} \\
\x{seq\_file\_name} is the name of the nucleotide or amino-acid sequence file in PHYLIP format.


\item \x{-d} (or \x{--datatype}) \x{data\_type}\\
\x{data\_type} is \x{nt} for nucleotide (default) and \x{aa} for amino-acid sequences.


\item \x{-q} (or \x{--sequential})\index{sequence format!interleaved}\index{sequence format!sequential} \\
Changes interleaved format (default) to sequential format.


\item \x{-n} (or \x{--multiple}) \x{nb\_data\_sets}\index{multiple data sets}\\
\x{nb\_data\_sets} is an integer giving the number of data sets to analyse.

\item \x{-p} (or \x{--pars})\\
Use a minimum parsimony starting tree. This option is taken into account when the '-u' option
is absent and when tree topology modifications are to be done.


\item \x{-b} (or \x{--bootstrap}) \x{int}\index{bootstrap} \\
\begin{itemize}  
\item \x{int} $>$  0: \x{int} is the number of bootstrap replicates.
\item \x{int} =  0: neither approximate likelihood ratio test nor bootstrap values are computed.
\item \x{int} = -1: approximate likelihood ratio test returning aLRT statistics.
\item \x{int} = -2: approximate likelihood ratio test returning Chi2-based parametric branch supports.
% \item \x{int} = -3: minimum of Chi2-based parametric and SH-like branch supports.
\item \x{int} = -4: SH-like branch supports alone.
\end{itemize}

\item \x{-m} (or \x{--model}) \x{model\_name}\index{substitution models!DNA}\index{substitution models!amino acids} \\
\x{model\_name} : substitution model name.
\begin{itemize}
\item {\it Nucleotide-based models}: \x{HKY85} (default) \x{| JC69 |  K80 | F81 | F84 | TN93 | GTR |
custom} \\  The \x{custom} option can be  used to define a  new substitution model. A  string of six
digits  identifies  the model.  For  instance,  000000 corresponds  to  F81  (or  JC69 provided  the
distribution of nucleotide  frequencies is uniform).  012345 corresponds to GTR.  This option can be
used for encoding any model that is a nested within GTR. See Section \ref{sec:submenus}. {\em NOTE:}
the  substitution  parameters  of  the  custom  model  will be  optimised  so  as  to  maximise  the
likelihood.  It is  possible  to  specify and  fix  (i.e., avoid  optimisation)  the  values of  the
substitution rates only through the PHYLIP-like interface.

\item {\it Amino-acid based models}: \x{LG} (default) \x{WAG | JTT | MtREV | Dayhoff | DCMut | RtREV
 | CpREV | VT | Blosum62 | MtMam | MtArt | HIVw |  HIVb | custom} \\
The \x{custom} option is  useful when one wants to use an amino-acid  substitution model that is not
available by  default in PhyML. The  symmetric part of the  rate matrix, as well  as the equilibrium
amino-acid frequencies, are  given in a file which name  is asked for by the  program. The format of
this file is described in section \ref{sec:customaa}.
\end{itemize}

\item \x{--aa\_rate\_file file\_name} \\
This option is compulsory when analysing amino-acid sequences under a `custom' model. \x{file\_name}
should provide a rate matrix and equilibrium amino acid in PAML format (see Section \label{sec:customaa}).

\item \x{-f e}, \x{m}, or ``\x{fA,fC,fG,fT}" \index{frequencies!nucleotide}\index{frequencies!amino-acid}\index{stationary frequencies}\\
Nucleotide or amino-acid frequencies.
\begin{itemize}
\item \x{e} : the character frequencies are determined as follows : 
\begin{itemize}
\item {\it Nucleotide sequences}: (Empirical) the equilibrium base frequencies are estimated by counting
                 the occurence of the different bases in the alignment. 
\item {\it Amino-acid sequences}: (Empirical) the equilibrium amino-acid frequencies are estimated by counting
                 the occurence of the different amino-acids in the alignment.
\end{itemize}  

\item \x{m} : the character frequencies are determined as follows : 
\begin{itemize}
\item {\it Nucleotide sequences}: (ML) the equilibrium base frequencies are estimated using maximum
  likelihood.
\item {\it Amino-acid sequences}: (Model) the equilibrium amino-acid frequencies are estimated using
                 the frequencies defined by the substitution model.
\end{itemize}

\item ``\x{fA,fC,fG,fT}" : only valid for nucleotide-based models. \x{fA}, \x{fC}, \x{fG} and \x{fT} are floating numbers that 
correspond to the frequencies of A, C, G and T respectively.
\end{itemize}  

\item \x{-t} (or \x{--ts/tv}) \x{ts/tv\_ratio} \index{$\kappa$}\index{ts/tv ratio}\\
\x{ts/tv\_ratio}: transition/transversion ratio. DNA sequences only. Can be a fixed positive value
(e.g., 4.0) or type \x{e} to get the maximum likelihood estimate.

\item \x{-v} (or \x{--pinv}) \x{prop\_invar}\index{proportion of invariants}\index{invariable sites} \\
\x{prop\_invar}: proportion of invariable sites. Can be a fixed value in the [0,1] range or type \x{e} to get the maximum likelihood estimate.

\item \x{-c} (or \x{--nclasses}) \x{nb\_subst\_cat}\index{gamma distribution (discrete)!number of categories} \\
\x{nb\_subst\_cat}: number of relative substitution rate categories. Default: \x{nb\_subst\_cat=4}. Must be a positive integer.

\item \x{-a} (or \x{--alpha}) \x{gamma} \index{gamma distribution (discrete)!shape parameter} \\
\x{gamma}: value of the gamma shape parameter. Can be a fixed positive value or e to get the maximum
likelihood estimate. The value of this parameter is estimated in the maximum likelihood framework by default.

\item \x{--use\_median} \index{gamma distribution (discrete)!mean vs. median} \\
The middle of each substitution rate class in the discrete gamma distribution is taken as the
median. The mean is used by default.

\item \x{-s} (or \x{--search}) \x{move}\index{NNI}\index{SPR} \\
Tree topology search operation option. Can be either \x{NNI} (default, fast) or \x{SPR} (a bit slower than \x{NNI}) or \x{BEST} (best of NNI and SPR search).

\item \x{-u} (or \x{--inputtree}) \x{user\_tree\_file}\index{input tree}\index{user tree} \\
\x{user\_tree\_file}: starting tree filename. The tree must be in Newick format.

\item \x{-o params}\index{optimisation!topology}\index{optimisation!substitution parameters} \\
This option focuses on specific parameter optimisation.
\begin{itemize}
\item \x{params=tlr}: tree topology (\x{t}), branch length (\x{l}) and substitution rate parameters (\x{r}) are optimised.
\item \x{params=tl}: tree topology and branch lengths are optimised.
\item \x{params=lr}: branch lengths and substitution rate parameters are optimised.
\item \x{params=l}: branch lengths are optimised.
\item \x{params=r}: substitution rate parameters are optimised.
\item \x{params=n}: no parameter is optimised.
\end{itemize}

\item \x{--rand\_start}\index{random tree} \\
This option sets the initial tree to random. It is only valid if SPR searches are to be performed.

\item \x{--n\_rand\_starts num} \\
\x{num} is the number of initial random trees to be used. It is only valid if SPR searches are to be performed.

\item \x{--r\_seed num}\index{random number} \\
\x{num} is the seed used to initiate the random number generator. Must be an integer.

\item \x{--print\_site\_lnl}\index{likelihood!print site likelihood} \\
Print the likelihood for each site in file *\_phyml\_lk.txt.

\item \x{--print\_trace} \\
Print each phylogeny explored during the tree search process in file *\_phyml\_trace.txt.

\item \x{--run\_id ID\_string}\index{run ID} \\
Append the string ID\_string at the end of each PhyML output file. This option may be useful when
running simulations involving PhyML. It can also be used to `tag' multiple analysis of the same data
set with various program settings.

\end{itemize}

  
\subsection{Parallel  bootstrap}\label{sec:parallel_bootstrap}\index{MPI}\index{bootstrap!parallel}

Bootstrapping is  a highly  parallelizable task. Indeed,  bootstrap replicates are  independent from
each other.   Hence, each bootstrap sample can  be analysed separately. Modern  computers often have
more than one CPU. Each CPU can therefore be used to process a bootstrap sample. Using this parallel
strategy, performing  $R$ bootstrap replicates  on $C$ CPUs  `costs' the same amount  of computation
time as processing $R  \times C$ bootstrap replicates on a single CPU.  In  other words, for a given
number of replicates, the computation time is divided by $R$ compared to the non-parallel approach.

PhyML sources  must be compiled with  specific options to turn  on the parallel  option (see Section
\ref{sec:MPI}). Once  the binary file (\x{phyml})  has been generated, running  a bootstrap analysis
with, say 100 replicates on 2 CPUs, can be done by typing the following command-line:
\begin{verbatim}
mpd &;
mpirun -np 2 ./phyml -i seqfile -b 100;
\end{verbatim} 
The  first command  launches  the mpi  daemon  while the  second launches  the  analysis. Note  that
launching the daemon needs to be done only once.  The output files are similar to the ones generated
using the standard, non-parallel, analysis (see Section \ref{sec:input_output}). Note that running 
the program in batch mode, i.e.:
\begin{verbatim}
mpirun -np 2 ./phyml -i seqfile -b 100 &
\end{verbatim} 
will probably NOT work. I do not know how to run a mpi process in batch mode yet. Suggestions welcome...
Also, at the moment, the number of bootstrap replicates must be a multiple of the number of CPUs
required in the mpirun command.

\section{Inputs / outputs}\label{sec:input_output}

PhyML reads data from standard text files,  without the need for any particular file name extension.

\subsection{Sequence formats}

\begin{figure}
\begin{small}
\begin{Verbatim}[frame=single, label=PHYLIP interleaved, samepage=true, baselinestretch=0.5]
5 80
seq1  CCATCTCACGGTCGGTACGATACACCKGCTTTTGGCAGGAAATGGTCAATATTACAAGGT
seq2  CCATCTCACGGTCAG---GATACACCKGCTTTTGGCGGGAAATGGTCAACATTAAAAGAT
seq3  RCATCTCCCGCTCAG---GATACCCCKGCTGTTG????????????????ATTAAAAGGT
seq4  RCATCTCATGGTCAA---GATACTCCTGCTTTTGGCGGGAAATGGTCAATCTTAAAAGGT
seq5  RCATCTCACGGTCGGTAAGATACACCTGCTTTTGGCGGGAAATGGTCAAT????????GT

ATCKGCTTTTGGCAGGAAAT
ATCKGCTTTTGGCGGGAAAT
AGCKGCTGTTG?????????
ATCTGCTTTTGGCGGGAAAT
ATCTGCTTTTGGCGGGAAAT
\end{Verbatim}
\begin{Verbatim}[frame=single, label=PHYLIP sequential, samepage=true, baselinestretch=0.5]
5 40
seq1  CCATCTCANNNNNNNNACGATACACCKGCTTTTGGCAGG
seq2  CCATCTCANNNNNNNNGGGATACACCKGCTTTTGGCGGG
seq3  RCATCTCCCGCTCAGTGAGATACCCCKGCTGTTGXXXXX
seq4  RCATCTCATGGTCAATG-AATACTCCTGCTTTTGXXXXX
seq5  RCATCTCACGGTCGGTAAGATACACCTGCTTTTGxxxxx
\end{Verbatim}
\end{small}
\label{fig:align_tree}
\caption{\bf PHYLIP interleaved and sequential formats.}
\end{figure}



\begin{figure}
\begin{small}
\begin{Verbatim}[frame=single, label=Nexus nucleotides, samepage=true, baselinestretch=0.5]
[ This is a comment ]
#NEXUS
BEGIN DATA;
DIMENSIONS NTAX=10 NCHAR=20;
FORMAT DATATYPE=DNA;
MATRIX
tax1       ?ATGATTTCCTTAGTAGCGG
tax2       CAGGATTTCCTTAGTAGCGG
tax3       ?AGGATTTCCTTAGTAGCGG
tax4       ?????????????GTAGCGG
tax5       CAGGATTTCCTTAGTAGCGG
tax6       CAGGATTTCCTTAGTAGCGG
tax7       ???GATTTCCTTAGTAGCGG
tax8       ????????????????????
tax9       ???GGATTTCTTCGTAGCGG
tax10      ???????????????AGCGG;
END;	                      	                                      	                                      	                  
\end{Verbatim}
\end{small}

\begin{small}
\begin{Verbatim}[frame=single, label=Nexus digits, samepage=true, baselinestretch=0.5]
[ This is a comment ]
#NEXUS
BEGIN DATA;
DIMENSIONS NTAX=10 NCHAR=20;
FORMAT DATATYPE=STANDARD SYMBOLS="0 1 2 3";
MATRIX
tax1       ?0320333113302302122
tax2       10220333113302302122
tax3       ?0220333113302302122
tax4       ?????????????2302122
tax5       10220333113302302122
tax6       10220333113302302122
tax7       ???20333113302302122
tax8       ????????????????????
tax9       ???22033313312302122
tax10      ???????????????02122;
END;	                      	                                      	                                      	                  
\end{Verbatim}
\end{small}

\begin{small}
\begin{Verbatim}[frame=single, label=Nexus digits, samepage=true, baselinestretch=0.5]
[ This is a comment ]
#NEXUS
BEGIN DATA;
DIMENSIONS NTAX=10 NCHAR=20;
FORMAT DATATYPE=STANDARD SYMBOLS="00 01 02 03";
MATRIX
tax1       ??00030200030303010103030002030002010202
tax2       0100020200030303010103030002030002010202
tax3       ??00020200030303010103030002030002010202
tax4       ??????????????????????????02030002010202
tax5       0100020200030303010103030002030002010202
tax6       0100020200030303010103030002030002010202
tax7       ??????0200030303010103030002030002010202
tax8       ????????????????????????????????????????
tax9       ??????0202000303030103030102030002010202
tax10      ??????????????????????????????0002010202;
END;	                      	                                      	                                      	                  
\end{Verbatim}
\end{small}
\caption{\bf NEXUS formats.}\label{fig:nexus}
\end{figure}


Alignments   of   DNA   or   protein   sequences   must  be   in   PHYLIP\index{PHYLIP}   or   NEXUS
\cite{maddison97}\index{NEXUS} sequential\index{sequential} or interleaved\index{interleaved} format
(Figures \ref{fig:align_tree}  and \ref{fig:nexus}).  For  PHYLIP formated sequence  alignments, the
first line of  the input file contains the number  of species and the number  of characters, in free
format, separated by blank characters.  One slight difference with PHYLIP format deals with sequence
name  lengths.  While PHYLIP  format limits  this length  to ten  characters, PhyML  can read  up to
hundred  character long sequence  names.  Blanks  and the  symbols ``(),:''  are not  allowed within
sequence names because  the Newick tree format  makes special use of these  symbols.  Another slight
difference with  PHYLIP format is  that actual sequences  must be separated  from their names  by at
least one blank character.

A PHYLIP input sequence file  may also display more than a single data set.  Each of these data sets
must  be  in  PHYLIP   format  and  two  successive  alignments  must  be   separated  by  an  empty
line. Processing multiple  data sets requires to toggle  the `\x{M}' option in the  {\em Input Data}
sub-menu or use the `\x{-n}' command line option  and enter the number of data sets to analyse.  The
multiple  data set  option can  be  used to  process re-sampled  data  that were  generated using  a
non-parametric  procedure such  as  cross-validation or  jackknife  (a bootstrap  option is  already
included in PhyML).  This  option is also useful in multiple gene studies,  even if fitting the same
substitution model to all data sets may not be suitable.

PhyML can  also process alignments in  NEXUS format. Although not  all the options  provided by this
format are supported  by PhyML, a few specific  features are exploited.  Of course,  this format can
handle nucleotide and  protein sequence alignments in sequential or interleaved  format.  It is also
possible  to  use custom  alphabets,  replacing  the standard  4-state  and  20-state alphabets  for
nucleotides and amino-acids respectively. Examples of  a 4-state custom alphabet are given in Figure
\ref{fig:nexus}. Each state must  here correspond to one digit or more. The set  of states must be a
list of consecutive  digits starting from 0.  For instance,  the list ``0, 1, 3, 4''  is not a valid
alphabet. Each  state in the  symbol list  must be separated  from the next  one by a  space. Hence,
alphabets with up to  100 states can be easily defined by using  two-digit number, starting with 00,
up  to 99.   Most importantly,  this feature  gives the  opportunity to  analyse data  sets  made of
presence/absence   character   states   (use   the   \texttt{symbols=``0  1''}   option   for   such
data).\index{binary characters} Alignments made of custom-defined states will be processed using the
Jukes and Cantor model.   Other options of the program (e.g., number  of rate classes, tree topology
search algorithm) are freely configurable.



\subsubsection{Gaps and ambiguous characters}

Gaps correspond to  the `\x{-}' symbol.  They are systematically treated  as unknown characters ``on
the grounds  that we  don't know what  would be  there if something  were there''  (J.  Felsenstein,
PHYLIP main documentation).   The likelihood at these  sites is summed over all  the possible states
(i.e.,  nucleotides  or   amino  acids)  that  could  actually  be   observed  at  these  particular
positions. Note however that  columns of the alignment that display only  gaps or unknown characters
are simply discarded because  they do not carry any phylogenetic information  (they are equally well
explained  by any  model).  PhyML  also handles  ambiguous characters  such as  $R$ for  $A$  or $G$
(purines) and $Y$ for $C$  or $T$ (pyrimidines).  Tables \ref{tab:ambigu_nt} and \ref{tab:ambigu_aa}
give the list of valid characters/symbols and the corresponding nucleotides or amino acids.

\begin{table}
\begin{center}
\begin{tabular}{lr|lr}
\hline
Character & Nucleotide &   Character & Nucleotide \\               
\hline                                   
$A$       & Adenosine &     $Y$       & $C$ or $T$ \\                 
$G$       & Guanine &       $K$       & $G$ or $T$ \\               
$C$       & Cytosine &      $B$       & $C$ or $G$ or $T$\\         
$T$       & Thymine &       $D$       & $A$ or $G$ or $T$ \\        
$U$       & Uracil (=$T$) & $H$       & $A$ or $C$ or $T$ \\        
$M$       & $A$ or $C$ &    $V$       & $A$ or $C$ or $G$ \\        
$R$       & $A$ or $G$ &    $-$ or $N$ or $X$ or $?$ & unknown  \\  
$W$       & $A$ or $T$ &    & (=$A$ or $C$ or $G$ or $T$)\\         
$S$       & $C$ or $G$ &   & \\
\hline
\end{tabular}
\end{center}
\caption{{\bf List of valid characters in DNA sequences and the corresponding nucleotides.}}\label{tab:ambigu_nt}
\end{table}
\begin{table}
\begin{center}
\begin{tabular}{lr|lr}
\hline
Character & Amino-Acid & Character & Amino-Acid \\
\hline
$A$       & Alanine &         $L$       & Leucine \\             
$R$       & Arginine &        $K$       & Lysine \\              
$N$ or $B$& Asparagine &      $M$       & Methionine \\          
$D$       & Aspartic acid &   $F$       & Phenylalanine \\       
$C$       & Cysteine &        $P$       & Proline \\             
$Q$ or $Z$& Glutamine &       $S$       & Serine \\              
$E$       & Glutamic acid &   $T$       & Threonine \\           
$G$       & Glycine &         $W$       & Tryptophan \\          
$H$       & Histidine &       $Y$       & Tyrosine \\            
$I$       & Isoleucine &      $V$       & Valine \\              
$L$       & Leucine &         $-$ or $X$ or $?$ & unknown \\     
$K$       & Lysine &          & (can be any amino acid) \\       
\hline
\end{tabular}
\end{center}
\caption{{\bf List of valid characters in protein sequences and the corresponding amino acids.}}\label{tab:ambigu_aa}
\end{table}

\subsection{Tree format}

PhyML can  read one or  several phylogenetic trees  from an input  file.  This option  is accessible
through the  {\em Tree Searching} sub  menu or the `\x{-u}'  argument from the  command line.  Input
trees are generally used as initial maximum  likelihood estimates to be subsequently adjusted by the
tree searching algorithm.   Trees can be either rooted or unrooted  and multifurcations are allowed.
Taxa names must, of course, match the corresponding sequence names.

\begin{figure}[h]
\begin{small}
\begin{minipage}{\textwidth}
\begin{verbatim}
((seq1:0.03,seq2:0.01):0.04,(seq3:0.01,(seq4:0.2,seq5:0.05):0.2):0.01);
((seq3,seq2),seq1,(seq4,seq5));
\end{verbatim}
\end{minipage}
\end{small}
\caption{{\bf Input trees}. The first tree (top) is rooted and has branch lengths. The second tree
  (bottom) is unrooted and does not have branch lengths.}
\label{fig:trees}\index{Newick format}
\end{figure}


\subsection{Multiple alignments and trees}\index{multiple data sets}

Single or  multiple sequence  data sets may  be used  in combination with  single or  multiple input
trees. When the number of data sets is one ($n_D = 1$) and there is only one input tree ($n_T = 1$),
then this tree is simply  used as input for the single data set analysis. When  $n_D = 1$ and $n_T >
1$,  each input tree  is used  successively for  the analysis  of the  single alignment.  PhyML then
outputs the tree  with the highest likelihood.  If $n_D > 1$ and  $n_T = 1$, the same  input tree is
used for the analysis  of each data set.  The last  combination is $n_D > 1$ and $n_T  > 1$. In this
situation, the  $i$-th tree in the input  tree file is used  to analyse the $i$-th  data set. Hence,
$n_D$ and $n_T$ must be equal here.


\subsection{Custom amino-acid rate model}\label{sec:customaa}

The custom amino-acid model of substitutions can be used to implement a model that is not hard-coded
in  PhyML.   This model  must  be  time-reversible.  Hence,  the  matrix  of  substitution rates  is
symmetrical. The format  of the rate matrix with the associated  stationary frequencies is identical
to the one used in PAML\index{PAML}. An example is given below:

\begin{center}
{\tiny
\begin{tabular}{p{0.33cm}p{0.33cm}p{0.33cm}p{0.33cm}p{0.33cm}p{0.33cm}p{0.33cm}p{0.33cm}p{0.33cm}p{0.33cm}p{0.33cm}p{0.33cm}p{0.33cm}p{0.33cm}p{0.33cm}p{0.33cm}p{0.33cm}p{0.33cm}p{0.33cm}p{0.33cm}}
% Ala & Arg & Asn & Asp & Cys & Gln & Glu & Gly & His & Ile & Leu & Lys & Met & Phe & Pro & Ser & Thr & Trp & Tyr & Val \\
  &&&&&&&&&&&&&&&&&&& \\
0.55 &  &&&&&&&&&&&&&&&&&& \\
0.51 & 0.64 &  &&&&&&&&&&&&&&&& \\
0.74 & 0.15 & 5.43 &  &&&&&&&&&&&&&&&& \\
1.03 & 0.53 & 0.27 & 0.03 &  &&&&&&&&&&&&&&& \\
0.91 & 3.04 & 1.54 & 0.62 & 0.10 &   &&&&&&&&&&&&&& \\
1.58 & 0.44 & 0.95 & 6.17 & 0.02 & 5.47 &  &&&&&&&&&&&&& \\
1.42 & 0.58 & 1.13 & 0.87 & 0.31 & 0.33 & 0.57 &  &&&&&&&&&&&& \\
0.32 & 2.14 & 3.96 & 0.93 & 0.25 & 4.29 & 0.57 & 0.25 &  &&&&&&&&&&& \\
0.19 & 0.19 & 0.55 & 0.04 & 0.17 & 0.11 & 0.13 & 0.03 & 0.14 &  &&&&&&&&&& \\
0.40 & 0.50 & 0.13 & 0.08 & 0.38 & 0.87 & 0.15 & 0.06 & 0.50 & 3.17 &  &&&&&&&&& \\
0.91 & 5.35 & 3.01 & 0.48 & 0.07 & 3.89 & 2.58 & 0.37 & 0.89 & 0.32 & 0.26 &  &&&&&&&& \\
0.89 & 0.68 & 0.20 & 0.10 & 0.39 & 1.55 & 0.32 & 0.17 & 0.40 & 4.26 & 4.85 & 0.93 &  &&&&&&& \\
0.21 & 0.10 & 0.10 & 0.05 & 0.40 & 0.10 & 0.08 & 0.05 & 0.68 & 1.06 & 2.12 & 0.09 & 1.19 &  &&&&&& \\
1.44 & 0.68 & 0.20 & 0.42 & 0.11 & 0.93 & 0.68 & 0.24 & 0.70 & 0.10 & 0.42 & 0.56 & 0.17 & 0.16 &  &&&&& \\
3.37 & 1.22 & 3.97 & 1.07 & 1.41 & 1.03 & 0.70 & 1.34 & 0.74 & 0.32 & 0.34 & 0.97 & 0.49 & 0.55 & 1.61 &  &&&& \\
2.12 & 0.55 & 2.03 & 0.37 & 0.51 & 0.86 & 0.82 & 0.23 & 0.47 & 1.46 & 0.33 & 1.39 & 1.52 & 0.17 & 0.80 & 4.38 &  &&& \\
0.11 & 1.16 & 0.07 & 0.13 & 0.72 & 0.22 & 0.16 & 0.34 & 0.26 & 0.21 & 0.67 & 0.14 & 0.52 & 1.53 & 0.14 & 0.52 & 0.11 &  && \\
0.24 & 0.38 & 1.09 & 0.33 & 0.54 & 0.23 & 0.20 & 0.10 & 3.87 & 0.42 & 0.40 & 0.13 & 0.43 & 6.45 & 0.22 & 0.79 & 0.29 & 2.49 &   & \\
2.01 & 0.25 & 0.20 & 0.15 & 1.00 & 0.30 & 0.59 & 0.19 & 0.12 & 7.82 & 1.80 & 0.31 & 2.06 & 0.65 & 0.31 & 0.23 & 1.39 & 0.37 & 0.31 &   \\
\\
8.66 & 4.40 & 3.91 & 5.70 & 1.93 & 3.67 & 5.81 & 8.33 & 2.44 & 4.85 & 8.62 & 6.20 & 1.95 & 3.84 & 4.58 & 6.95 & 6.10 & 1.44 & 3.53 & 7.09  \\
\end{tabular}
}
\end{center}

The  entry  on the  $i$-th  row  and  $j$-th  column of  this  matrix  corresponds  to the  rate  of
substitutions between  amino-acids $i$  and $j$.   The last line  in the  file gives  the stationary
frequencies and must be separated from the rate  matrix by one line. The ordering of the amino-acids
is alphabetical,  i.e, Ala, Arg, Asn, Asp,  Cys, Gln, Glu, Gly,  His, Ile, Leu, Lys,  Met, Phe, Pro,
Ser, Thr, Trp, Tyr and Val.

\subsection{Output files}

\begin{table}
Sequence file name~: `{\x seq}'\\
\begin{center}
\begin{tabular}{ll}
\hline
Output file name & Content \\
\hline
\x{seq\_phyml\_tree.txt} & ML tree\\
\x{seq\_phyml\_stats.txt} &  ML model parameters\\
\x{seq\_phyml\_boot\_trees.txt} & ML trees -- bootstrap replicates\\
\x{seq\_phyml\_boot\_stats.txt} & ML model parameters -- bootstrap replicates \\
\x{seq\_phyml\_rand\_trees.txt} & ML trees -- multiple random starts\\
\hline
\end{tabular}
\end{center}
\caption{{\bf Standard output files}}\label{tab:output}
\end{table}

Table  \ref{tab:output} presents  the list  of files  resulting from  an analysis.   Basically, each
output file  name can be divided into  three parts.  The first  part is the sequence  file name, the
second part corresponds to  the extension `\x{\_phyml\_}' and the third part  is related to the file
content.  When launched with the default options,  PhyML only generates two files: the tree file and
the model parameter file.   The estimated maximum likelihood tree is in  standard Newick format (see
Figure  \ref{fig:trees}).  The  model  parameters file,  or  statistics file,  displays the  maximum
likelihood estimates of the substitution model  parameters, the likelihood of the maximum likelihood
phylogenetic model, and  other important information concerning the settings  of the analysis (e.g.,
type of data, name of the substitution model, starting tree, etc.).  Two additional output files are
created if  bootstrap supports were  evaluated.  These files  simply contain the  maximum likelihood
trees  and  the  substitution  model  parameters  estimated from  each  bootstrap  replicate.   Such
information can be used to estimate sampling errors around each parameter of the phylogenetic model.
When the random  tree option is turned on,  the maximum likelihood trees estimated  from each random
starting trees are printed in a separate tree file (see last row of Table \ref{tab:output}).




\section{Recommendations on program usage.}

The choice of the  tree searching algorithm among those provided by PhyML  is generally a tough one.
The  fastest option  relies  on local  and simultaneous  modifications  of the  phylogeny using  NNI
moves. More  thorough explorations of  the space  of topologies are  also available through  the SPR
options.  As these  two classes of tree topology moves involve  different computational burdens, it
is important to determine which option is the most suitable for the type of data set or analysis one
wants to perform. Below is a list of recommendations for typical phylogenetic analyses.

\begin{enumerate}
\item {\em Single data set, unlimited computing time.} The best option here is probably to use a SPR
search (i.e., straight SPR of best of SPR and NNI).  If the focus is on estimating the relationships
between species,  it is a good  idea to use  more than one starting  tree to decrease the  chance of
getting stuck  in a  local maximum of  the likelihood  function.  Using NNIs  is appropriate  if the
analysis does not mainly focus on  estimating the evolutionary relationships between species (e.g. a
tree is needed to  estimate the parameters of codon-based models later  on).  Branch supports can be
estimated using bootstrap and approximate likelihood ratios.

\item {\em  Single data set, restricted  computing time.}  The  three tree searching options  can be
used depending on  the computing time available and the  size of the data set.   For small data sets
(i.e., $<$ 50 sequences),  NNI will generally perform well provided that  the phylogenetic signal is
strong.  It  is relevant  to estimate a  first tree  using NNI moves  and examine  the reconstructed
phylogeny in order to have a rough idea  of the strength of the phylogenetic signal (the presence of
small internal  branch lengths  is generally  considered as a  sign of  a weak  phylogenetic signal,
specially when  sequences are  short).  For larger  data sets  ($>$ 50 sequences),  a SPR  search is
recommended if there  are good evidence of  a lack of phylogenetic signal.   Bootstrap analysis will
generally  involve  large  computational  burdens.   Estimating branch  supports  using  approximate
likelihood ratios therefore provides an interesting alternative here.

\item {\em  Multiple data  sets, unlimited computing  time.} Comparative genomic  analyses sometimes
rely on building phylogenies from the analysis of  a large number of gene families.  Here again, the
NNI option is the most  relevant if the focus is not on recovering the  most accurate picture of the
evolutionary relationships  between species.   Slower SPR-based heuristics  should be used  when the
topology of the tree is an important parameter of the analysis (e.g., identification of horizontally
transferred genes using phylogenetic tree comparisons).   Internal branch support is generally not a
crucial parameter of the multiple data  set analyses. Using approximate likelihood ratio is probably
the best choice here.

\item {\em Multiple data sets, limited computing time.}  The large amount of data to be processed in
a  limited time  generally  requires  the use  of  the fastest  tree  searching  and branch  support
estimation methods Hence,  NNI and approximate likelihood ratios rather  than SPR and non-parametric
bootstrap are generally the most appropriate here.
\end{enumerate}
 
Another important  point is the  choice of the  substitution model. While default  options generally
provide acceptable results, it is often warranted to perform a pre-analysis in order to identify the
best-fit substitution model.  This pre-analysis can be done using popular software such as Modeltest
\cite{posada98} or ProtTest  \cite{abascal05} for instance.  These programs  generally recommend the
use of a discrete gamma distribution to model the substitution process as variability of rates among
sites is a common  feature of molecular evolution.  The choice of the number  of rate classes to use
for this  distribution is  also an important  one. While the  default is  set to four  categories in
PhyML, it is  recommended to use larger number  of classes if possible in order  to best approximate
the  patterns of rate  variation across  sites \cite{galtier04}.   Note however  that run  times are
directly proportional to  the number of classes  of the discrete gamma distribution.   Here again, a
pre-analysis with the  simplest model should help the  user to determine the number  of rate classes
that represents the best trade-off between computing time and fit of the model to the data.


\section{Other programs in the PhyML package}

As well as being a computer program that builds maximum likelihood phylogenies, PhyML is software
package that provides tools to tackle different problems. Installing these programs and processing
data sets is explained is the following sections.

\subsection{PhyTime (Guindon, {\it Mol. Biol. Evol.} 2010)}\index{PhyTime} PhyTime is a program that
estimates node  ages and substitution rates  using a fast Bayesian  approach.  It relies  on a Gibbs
sampler  which  outperforms  the  ``standard''  Metropolis-Hastings algorithm  implemented  in  most
phylogenetic  softwares that  aim at  estimating node  ages.  The  details and  performance  of this
approach are described in the following  paper: Guindon S. ``Bayesian estimation of divergence times
from large data sets'', {\it Mol. Biol. Evol.}, 2010.

\subsubsection{Installing PhyTime}

Compiling PhyTime is straightforward on Unix-like  machines (i.e., linux and MacOS systems). PhyTime
is not readily available for Windows machines but  compilation should be easy on this system too. In
the `phyml' directory, where the `src/' and `doc/' directories stand, enter the following commands:
{\setlength{\baselineskip}{0.5\baselineskip}
\begin{verbatim}
./configure --enable-times;
make clean;
make;
\end{verbatim} } This set of commands generates  a binary file called \x{phytime} which can be found
  in the `src/' directory.

\subsubsection{Running  PhyTime} Passing  options and  running  PhyTime on  your data  set is  quite
similar to running  PhyML in commmand-line mode.  The main differences between the  two programs are
explained below:
\begin{itemize}
\item PhyTime takes as mandatory input a {\em rooted} phylogenetic tree.  Hence, the `\x{-u}' option
must be used. Also, unlike PhyML, PhyTime does not modify the tree topology. Hence, the options that
go with the '\x{-s}' command do not alter the input tree topology.
\item  PhyTime  needs  an input  file  giving  information  about  calibration nodes.   The  command
`\x{--calibration=}' followed by the name of  the file containing the calibration node information is
mandatory. The content of that file should look as follows:

\begin{figure}[h]
\begin{small}
\begin{Verbatim}[frame=single, label=Calibration node file, samepage=true, baselinestretch=0.5]
Dugong_dugon Procavia_capensis Elephantidae | -65 -54
Equus_sp. Ceratomorpha | -58 -54
Cercopithecus_solatus Macaca_mulatta Hylobates_lar Homo_sapiens | -35 -25
Lepus_crawshayi Oryctolagus_cuniculus Ochotona_princeps | -90 -37
Marmota_monax Aplodontia_rufa | -120 -37 
Dryomys_nitedula Glis_glis | -120 -28.5
@root@ | -100 -120
\end{Verbatim}
\end{small}
\end{figure}

Every row in this file lists a set of species found in the user tree (i.e., a clade). This list of
taxa is followed by the character `\x{|}' and two real numbers corresponding to the lower and upper
bounds of the calibration interval for the node at the root of the clade. In the example given here,
the clade grouping the three taxa ``Dugong\_dugon'', ``Procavia\_capensis'' and  ``Elephantida'' has
-65 as lower bound and -54 as upper bound. Note also that the node corresponding to the root of the
whole tree has a specific label: `\x{@root@}'.
\end{itemize}

A typical PhyTime command-line should look like the following:

\begin{Verbatim}[fontsize=\small]
./phytime -i seqname -u treename --calibration=calibration_file -m GTR -c 8
\end{Verbatim}

Assuming the file `\x{seqname}' contains DNA  sequences in PHYLIP or NEXUS format, `\x{treename}' is
the rooted  input tree in NEXUS  format and `\x{calibration\_file}'  is a set of  calibration nodes,
PhyTime will  estimate the  posterior distribution of  node times  and substitution rates  under the
assumption that the  substitution process follows a GTR  model with 8 classes of rates  in the Gamma
distribution of  rates across sites. The  model parameter values  are estimated by a  Gibbs sampling
technique. This algorithm tries  diferent sets of model parameter values and  only conserve the most
probable ones. By default, $10^8$ sets of  parameters are collected. These values are recorder every
$10^4$ sample. At the  moment, the total sample size and the frequency  at which these sample values
are recorded can not be modified by the user.

\subsubsection{PhyTime output}

The program PhyTime generates two output  files. The first file, called `\x{phytime.XXXX} where XXXX
is a randomly generated  integer, lists the node times and branch  relative rates sampled during the
estimation  process. It  also gives  the sampled  values  for other  parameter values,  such as  the
autocorrelation of  rates (parameter `Nu'), the  overall rates of evolution  (parameter `Clock') and
the  average of the  relative substitution  rates (parameter  `MeanRate'). This  output file  can be
analysed       with      the       program       Tracer      from       the      BEAST       package
(\url{http://beast.bio.ed.ac.uk/Main_Page}). The second  file is called `\x{phytime.XXXX.trees}'. It
is the list  of trees that were  collected during the estimation process,  i.e., phylogenies sampled
from the posterior density  of trees.  This file can be processed  using the software TreeAnnotator,
also part of the BEAST  package (see \url{http://beast.bio.ed.ac.uk/Main_Page}) in order to generate
confidence sets for the node time estimates.

% \subsection{PhyGeo (Guindon, {\it in preparation})}\index{PhyGeo}



\section{Frequently asked questions}  

\begin{enumerate}
\item {\it PhyML crashes before reading the sequences. What's wrong ?}\\
\begin{itemize}
\item The format of your sequence file is not recognized by PhyML. See Section \ref{sec:input_output}
\item The carriage return characters in your sequence files are not recognized by PhyML. You must
  make sure that your sequence file is a plain text file, with standard carriage return characters (i.e.,
  corresponding to ``$\backslash$\x{n}'', or ``$\backslash$\x{r}'')
\end{itemize}
\item {\it The program crashes after reading the sequences. What's wrong ?}\\
\begin{itemize}
\item You analyse protein sequences and did not enter the \x{-d aa} option in the command-line.
\item The format of your sequence file is not recognized by PhyML. See Section \ref{sec:input_output}
\end{itemize}
\item {\it Does PhyML handle outgroup sequences ?}\\
\begin{itemize}
\item No,  PhyML does  not make  any difference between  outgroup and  ingroup sequences.   The best
solution to take into account outgroup sequences is to run two separate analysis. The first analysis
should be  conducted on the set  of aligned sequences  {\it excluding} the outgroup  sequences. This
data set is  used to estimate the ingroup  phylogeny. The second analysis includes the  whole set of
sequences. The tree corresponds to the ingroup+outgroup phylogeny. The third step is to position the
root on the ingroup phylogeny using  the ingroup+outgroup phylogeny. The advantage of this technique
is to  avoid long-branch attraction  in the phylogeny  estimation due to distantly  related outgroup
sequences.
\end{itemize}
\item {\it Does PhyML estimate clock-constrained trees ?}\\
\begin{itemize}
\item No, PhyML cannot  do that at the moment. However, future releases  of the program will include
this feature.
\end{itemize}
\item {\it Can PhyML analyse partitioned data, such as multiple gene sequences ?}\\
\begin{itemize}
\item We are currently  working on this topic.  Future releases of  the program will provide options
to estimate  trees from phylogenomic data sets,  with the opportunity to  use different substitution
models on  the different data partitions (e.g.,  different genes). PhyML will  also include specific
algorithms to search the space of tree topologies for this type of data.
\end{itemize}
\end{enumerate}



\section{Acknowledgements}  
The development of PhyML since 2000 has been supported by the Centre National de la Recherche
Scientifique (CNRS) and the Minist\`ere de l'\'Education Nationale.

% \bibliographystyle{/home/guindon/latex/biblio/OUPnum}
\bibliographystyle{/Users/guindon/latex/biblio/nature/naturemag}
\bibliography{/Users/guindon/latex/biblio/ref}

\printindex
\end{document}